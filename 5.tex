\section{Jeffreys Priors.}

\begin{align*}
    g(\theta) &\propto \sqrt{I(\theta)}\\
    I(\theta) &= \left< \left[\frac{\d}{\d\theta} \ln(L(x|\theta))\right]^2 \right> = \int dx L(x|\theta) \left[\frac{\d}{\d\theta} \ln(L(x|\theta))\right]^2\\
\end{align*}

\begin{enumerate}[label=\textbf{\Alph*}.]
    \item Consider a measurement in which we flip a single coin once, and want to estimate the probability $p$ for the coin coming up heads. Derive the Jeffreys prior $g(p)$ in this case.

    Use the formula, where the likelihood of each hypothesis is binomial, with heads = ``success'', i.e. $L(n\text{ heads in } N \text{ trials}|p) = \frac{N!}{n!(N-n)!} p^n (1-p)^{N-n}$.

    $L(n|N, p) = \frac{N!}{n!(N-n)!} p^n (1-p)^{N-n}$

    \begin{align*}
        \frac{\d}{\d p} \ln(L(n|N, p)) &= \frac{\d}{\d p} \ln\left(\frac{N!}{n!(N-n)!} p^n (1-p)^{N-n}\right) \\
        &= \frac{\d}{\d p} \ln\left(\frac{N!}{n!(N-n)!}\right) + n\ln(p) + (N-n)\ln(1-p) \\
        &= 0 + \frac{n}{p} - \frac{N-n}{1-p} \\
    \end{align*}

    Here we flip a single coin once, so $N=1, n \in \{0,1\}$
    \begin{align*}
        L(n|p) &= p^n (1-p)^{1-n} \\
        \frac{\d}{\d p} \ln(L(n|p)) &= \frac{n}{p} - \frac{1-n}{1-p} \\
    \end{align*}

    \begin{align*}
        I(p) &=  \left< \left[\frac{\d}{\d p} \ln(L(n|p))\right]^2 \right> \\
        &= \left< \left[\frac{n}{p} - \frac{1-n}{1-p}\right]^2 \right> \\
        &= \left< \frac{(n-p)^2}{p^2(1-p)^2} \right> \\
        &= \sum_n L(n|p) \frac{(n-p)^2}{p^2(1-p)^2} \\
        &= L(0|p) \frac{(-p)^2}{p^2(1-p)^2} + L(1|p) \frac{(1-p)^2}{p^2(1-p)^2} \\
        &= (1-p) \frac{1}{(1-p)^2} + p \frac{1}{p^2} \\
        &= \frac{1}{1-p} + \frac{1}{p} \\
        &= \frac{1}{p(1-p)} \\
        g(p) &\propto \frac{1}{\sqrt{p(1-p)}} \\
    \end{align*}

    Find the constant of proportionality by normalizing:
    \begin{align*}
        \int_0^1 g(p)dp &= 1 \\
        \int_0^1 A \frac{1}{\sqrt{p(1-p)}} dp &= 1 \\
        A &= \frac{1}{\pi} \\
        \implies g(p) &= \frac{1}{\pi\sqrt{p(1-p)}} \\
    \end{align*}

    \item Suppose that you start with this prior, then flip the coin three times, yielding three heads. What is the probability that $p<0.5$?
    
    Prior: $g(p) = \frac{1}{\pi\sqrt{p(1-p)}}$

    Likelihood: $P(\text{3 heads}|p) = \prod_{i=1}^3 P(\text{heads}|p) = p^3$

    Probability of data:
    \begin{align*}
        P(\text{3 heads}) &= \int P(p) P(\text{3 heads}|p) dp \\
        &= \int \frac{1}{\pi\sqrt{p(1-p)}} p^3 dp \\
        &= \int_0^1 \frac{p^3}{\pi\sqrt{p(1-p)}} dp \\
        &= \frac{5}{16} \\
    \end{align*}

    Bayes's Theorem:
    \begin{align*}
        P(p|\text{3 heads}) &= \frac{P(p)P(\text{3 heads}|p)}{P(\text{3 heads})} \\
        &= \frac{\frac{1}{\pi\sqrt{p(1-p)}} p^3 }{\frac{5}{16}} \\
        &= \frac{16p^3}{5\pi\sqrt{p(1-p)}} \\
    \end{align*}

    So the probability that $p<0.5$ is
    \begin{align*}
        P &= \int_0^{0.5} P(p|\text{3 heads}) dp \\
        &= \int_0^{0.5} \frac{16p^3}{5\pi\sqrt{p(1-p)}} dp \\
        &= \frac{15\pi - 44}{30\pi} \\
        &\approx 0.033 \\
    \end{align*}

    \item Suppose $p = \psi^4$. Derive the Jeffreys prior for $\psi$, starting with the likelihood for a single coin flip expressed as a function of $\psi$.
    
    Use the Jeffreys prior formula again, where the likelihood is  our binomial from before, with $p = \psi^4$. Note that we take $\psi \in [0, 1]$, we could also use $\psi \in [-1, 1]$ but this would mean we have to worry about absolute values, which are yucky.

    \begin{align*}
        L(n|N, \psi) &= L(n|N, p=\psi^4) \\
        &= \frac{N!}{n!(N-n)!} \psi^{4n} (1-\psi^4)^{N-n} \\
    \end{align*}
    \begin{align*}
        L(n|N=1, \psi) &= \psi^{4n} (1-\psi^4)^{1-n} \\
        \ln(L(n|\psi)) &= 4n\ln(\psi) + (1-n)\ln(1-\psi^4) \\
        \frac{\d}{\d \psi} \ln(L(n|\psi)) &= 4n\frac{1}{\psi} + (1-n)\frac{1}{1-\psi^4}(-4\psi^3) \\
        &= \frac{4n(1-\psi^4)}{\psi(1-\psi^4)} + \frac{(1-n)(-4\psi^4)}{\psi(1-\psi^4)} \\
        &= \frac{4(n-\psi^4)}{\psi(1-\psi^4)} \\
    \end{align*}

    \begin{align*}
        I(\psi) &=  \left< \left[\frac{\d}{\d \psi} \ln(L(n|\psi))\right]^2 \right> \\
        &= \left< \left[\frac{4(n-\psi^4)}{\psi(1-\psi^4)}\right]^2 \right> \\
        &= \left< \frac{16(n-\psi^4)^2}{\psi^2(1-\psi^4)^2} \right> \\
        &= \sum_n L(n|\psi) \frac{16(n-\psi^4)^2}{\psi^2(1-\psi^4)^2} \\
        &= L(0|\psi) \frac{16(0-\psi^4)^2}{\psi^2(1-\psi^4)^2} + L(1|\psi) \frac{16(1-\psi^4)^2}{\psi^2(1-\psi^4)^2} \\
        &= (1-\psi^4) \frac{16\psi^6}{(1-\psi^4)^2} + \psi^4 \frac{16}{\psi^2} \\
        &= \frac{16\psi^6}{1-\psi^4} +  \frac{16\psi^2(1-\psi^4)}{1-\psi^4} \\
        &= \frac{16\psi^2}{1-\psi^4} \\
        g(\psi) &\propto \frac{4\psi}{\sqrt{1-\psi^4}} \\
    \end{align*}

    Find the constant of proportionality by normalizing:
    \begin{align*}
        \int_0^1 g(\psi)d\psi &= 1 \\
        \int_0^1 A \frac{4\psi}{\sqrt{1-\psi^4}} d\psi &= 1 \\
        A &= \frac{1}{\pi} \\
        g(\psi) &= \frac{4\psi}{\pi\sqrt{1-\psi^4}} \\
    \end{align*}

    \item Demonstrate explicitly that if you take the Jeffreys prior for $\psi$ from Part C and do a change of variables to $p$, you get back the Jeffreys prior for $p$ that you found in part A. This will confirm that Jeffreys' procedure for generating priors encodes the same information for both of these parametrizations.

    \begin{align*}
        g(\psi) &= \frac{4\psi}{\pi\sqrt{1-\psi^4}} \\
        g(\psi) &= g(p)\left|\frac{dp}{d\psi}\right| \\
        &= \frac{1}{\pi\sqrt{p(1-p)}}\left|\frac{d}{d\psi}\psi^4\right| \\
        &= \frac{1}{\pi\sqrt{\psi^4(1-\psi^4)}}\left|4\psi^3\right| \\
        &= \frac{4}{\pi\sqrt{\psi^4(1-\psi^4)}}\psi^2|\psi| \\
        &= \frac{4\psi}{\pi\sqrt{(1-\psi^4)}} \\
    \end{align*}

    (we can drop the absolute value since we've defined $\psi$ to be positive)

    \newpage
    \item Finally, show that if you started with a uniform prior for $p$ and a uniform prior for $\psi$, then these priors are actually different after converting from one parametrization to another with a change of variables. Thus a uniform prior is not a Jeffreys prior for this problem.
    
    Uniform for $p$: $P(p) = 1, p \in [0,1]$.

    Change of variables from $p \to \psi$:

    \begin{align*}
        P(\psi) &= P(p) \left|\frac{dp}{d\psi}\right| \\
        &= (1) 4\psi^3 \\
        4\psi^3 &\neq \text{a uniform distribution.}
    \end{align*}

\end{enumerate}
